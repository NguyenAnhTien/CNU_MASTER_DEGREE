
\documentclass{article}

\usepackage{fancyhdr} % Required for custom headers
\usepackage{lastpage} % Required to determine the last page for the footer
\usepackage{extramarks} % Required for headers and footers
\usepackage[usenames,dvipsnames]{color} % Required for custom colors
\usepackage{graphicx} % Required to insert images
\usepackage{listings} % Required for insertion of code
\usepackage{courier} % Required for the courier font
\usepackage{lipsum} % Used for inserting dummy 'Lorem ipsum' text into the template
\usepackage{caption}
\usepackage{subcaption}
\usepackage{amsmath}

\graphicspath{ {../../../datasets/images/chapter_02/} }

% Margins
\topmargin=-0.45in
\evensidemargin=0in
\oddsidemargin=0in
\textwidth=6.5in
\textheight=9.0in
\headsep=0.25in

\linespread{1.1} % Line spacing

% Set up the header and footer
\pagestyle{fancy}
\lhead{\hmwkAuthorName} % Top left header
\chead{\hmwkClass\ (\hmwkClassInstructor\ \hmwkClassTime): \hmwkTitle} % Top center head
\rhead{\firstxmark} % Top right header
\lfoot{\lastxmark} % Bottom left footer
\cfoot{} % Bottom center footer
\rfoot{Page\ \thepage\ of\ \protect\pageref{LastPage}} % Bottom right footer
\renewcommand\headrulewidth{0.4pt} % Size of the header rule
\renewcommand\footrulewidth{0.4pt} % Size of the footer rule

\setlength\parindent{0pt} % Removes all indentation from paragraphs

%----------------------------------------------------------------------------------------
%	CODE INCLUSION CONFIGURATION
%----------------------------------------------------------------------------------------

\definecolor{MyDarkGreen}{rgb}{0.0,0.4,0.0} 
\lstloadlanguages{Matlab}
\lstset{language=Matlab,
        frame=single,
        basicstyle=\small\ttfamily,
        keywordstyle=[1]\color{Blue}\bf,
        keywordstyle=[2]\color{Purple},
        keywordstyle=[3]\color{Blue}\underbar,
        identifierstyle=,
        commentstyle=\usefont{T1}{pcr}{m}{sl}\color{MyDarkGreen}\small, 
        stringstyle=\color{Purple},
        showstringspaces=false,
        tabsize=5, 
        morekeywords={rand},
        morekeywords=[2]{on, off, interp},
        morekeywords=[3]{test},
        morecomment=[l][\color{Blue}]{...},
        numbers=left,
        firstnumber=1,
        numberstyle=\tiny\color{Blue},
        stepnumber=5
}

% Creates a new command to include a perl script, the first parameter is the filename of the script (without .pl), the second parameter is the caption
\newcommand{\matlabscript}[2]{
\begin{itemize}
\item[]\lstinputlisting[caption=#2,label=#1]{#1.m}
\end{itemize}
}

%----------------------------------------------------------------------------------------
%	DOCUMENT STRUCTURE COMMANDS
%	Skip this unless you know what you're doing
%----------------------------------------------------------------------------------------

% Header and footer for when a page split occurs within a problem environment
\newcommand{\enterProblemHeader}[1]{
\nobreak\extramarks{#1}{#1 continued on next page\ldots}\nobreak
\nobreak\extramarks{#1 (continued)}{#1 continued on next page\ldots}\nobreak
}

% Header and footer for when a page split occurs between problem environments
\newcommand{\exitProblemHeader}[1]{
\nobreak\extramarks{#1 (continued)}{#1 continued on next page\ldots}\nobreak
\nobreak\extramarks{#1}{}\nobreak
}

\setcounter{secnumdepth}{0} % Removes default section numbers
\newcounter{homeworkProblemCounter} % Creates a counter to keep track of the number of problems

\newcommand{\homeworkProblemName}{}
\newenvironment{homeworkProblem}[1][Problem \arabic{homeworkProblemCounter}]{ % Makes a new environment called homeworkProblem which takes 1 argument (custom name) but the default is "Problem #"
\stepcounter{homeworkProblemCounter} % Increase counter for number of problems
\renewcommand{\homeworkProblemName}{#1} % Assign \homeworkProblemName the name of the problem
\section{\homeworkProblemName} % Make a section in the document with the custom problem count
\enterProblemHeader{\homeworkProblemName} % Header and footer within the environment
}{
\exitProblemHeader{\homeworkProblemName} % Header and footer after the environment
}

\newcommand{\problemAnswer}[1]{ % Defines the problem answer command with the content as the only argument
\noindent\framebox[\columnwidth][c]{\begin{minipage}{0.98\columnwidth}#1\end{minipage}} % Makes the box around the problem answer and puts the content inside
}

\newcommand{\homeworkSectionName}{}
\newenvironment{homeworkSection}[1]{ % New environment for sections within homework problems, takes 1 argument - the name of the section
\renewcommand{\homeworkSectionName}{#1} % Assign \homeworkSectionName to the name of the section from the environment argument
\subsection{\homeworkSectionName} % Make a subsection with the custom name of the subsection
\enterProblemHeader{\homeworkProblemName\ [\homeworkSectionName]} % Header and footer within the environment
}{
\enterProblemHeader{\homeworkProblemName} % Header and footer after the environment
}

%----------------------------------------------------------------------------------------
%	NAME AND CLASS SECTION
%----------------------------------------------------------------------------------------

\newcommand{\hmwkTitle}{Assignment\ \#1}
\newcommand{\hmwkDueDate}{Monday,\ September\ 11,\ 2017}
\newcommand{\hmwkClass}{Pattern Recognition} % Course/class
\newcommand{\hmwkClassTime}{09:10am} % Class/lecture time
\newcommand{\hmwkClassInstructor}{shkim} % Teacher/lecturer
\newcommand{\hmwkAuthorName}{Tien Anh Nguyen} % Your name

%----------------------------------------------------------------------------------------
%	TITLE PAGE
%----------------------------------------------------------------------------------------

\title{
\vspace{2in}
\textmd{\textbf{\hmwkClass:\ \hmwkTitle}}\\
\normalsize\vspace{0.1in}\small{Due\ on\ \hmwkDueDate}\\
\vspace{0.1in}\large{\textit{\hmwkClassInstructor\ \hmwkClassTime}}
\vspace{3in}
}

\author{\textbf{\hmwkAuthorName}}
\date{} % Insert date here if you want it to appear below your name

%----------------------------------------------------------------------------------------

\begin{document}

\maketitle

%----------------------------------------------------------------------------------------
%	TABLE OF CONTENTS
%----------------------------------------------------------------------------------------

%\setcounter{tocdepth}{1} % Uncomment this line if you don't want subsections listed in the ToC

\newpage
\tableofcontents
\newpage

%----------------------------------------------------------------------------------------
%	PROBLEM 1
%----------------------------------------------------------------------------------------

% To have just one problem per page, simply put a \clearpage after each problem

\begin{homeworkProblem}
\matlabscript{exercise_01_03_01}{Exercise 1.3.1}
With P1 = \(\frac{1.0}{6}\) and P2 = \(\frac{5.0}{6}\): 
we have p11 = 0.0140, p21 = 0.0314.\\
With P1 = \(\frac{5.0}{6}\) and P2 = \(\frac{1.0}{6}\):
we have p12 = 0.0699, p22 = 0.0063 

\end{homeworkProblem}

%----------------------------------------------------------------------------------------

%----------------------------------------------------------------------------------------
%	PROBLEM 2
%----------------------------------------------------------------------------------------

\newpage
\begin{homeworkProblem}
\matlabscript{exercise_01_04_01}{Exercise 1.4.1}
With N = 500:
we have 
\begin{equation}
m\_hat = 
                    \begin{bmatrix}
                        2.0478 \\
                        -2.0288
                    \end{bmatrix}\\
;S\_hat = 
                    \begin{bmatrix}
                        1.0257 & 0.2411 \\
                        0.2411 & 0.3190
                    \end{bmatrix}
\end{equation}

With N = 5000:
we have 
\begin{equation}
m\_hat = 
                    \begin{bmatrix}
                        1.9796 \\
                        -2.0025
                    \end{bmatrix}
;S\_hat = 
                    \begin{bmatrix}
                        0.8830 & 0.1987 \\
                        0.1987 & 0.3077
                    \end{bmatrix}
\end{equation}
\end{homeworkProblem}

%----------------------------------------------------------------------------------------

%----------------------------------------------------------------------------------------
%	PROBLEM 3
%----------------------------------------------------------------------------------------
\newpage
\begin{homeworkProblem}
\matlabscript{exercise_01_04_02}{Exercise 1.4.2}
With 
\begin{equation}
S_1 = S_2 = S_3 = 
    \begin{bmatrix}
        0.8 & 0.2 & 0.1 \\
        0.2 & 0.8 & 0.2 \\
        0.1 & 0.2 & 0.8
    \end{bmatrix}
\neq sigma^2I
\end{equation}
Result:
err\_euclidean = 0.1121
err\_mahalanobis = 0.1021
err\_bayesian = 0.0971

\end{homeworkProblem}
%----------------------------------------------------------------------------------------

%----------------------------------------------------------------------------------------
%       PROBLEM 4
%----------------------------------------------------------------------------------------
\newpage
\begin{homeworkProblem}
\matlabscript{exercise_01_04_03}{Exercise 1.4.3}
With \[P_1 = 1/2, P_2 = P_3 = 1/4\]
Result:
err\_euclidean = 0.0630
err\_mahalanobis = 0.0680
err\_bayesian = 0.0590
\end{homeworkProblem}
%----------------------------------------------------------------------------------------

%----------------------------------------------------------------------------------------
%       PROBLEM 5
%----------------------------------------------------------------------------------------
\newpage
\begin{homeworkProblem}
\matlabscript{exercise_01_04_04}{Exercise 1.4.4}
\end{homeworkProblem}
%----------------------------------------------------------------------------------------

%----------------------------------------------------------------------------------------
%       PROBLEM 6
%----------------------------------------------------------------------------------------
\newpage
\begin{homeworkProblem}
\matlabscript{exercise_01_07_01}{Exercise 1.7.1}
\end{homeworkProblem}
%----------------------------------------------------------------------------------------

%----------------------------------------------------------------------------------------
%       PROBLEM 7
%----------------------------------------------------------------------------------------
\newpage
\begin{homeworkProblem}
\matlabscript{exercise_01_07_02}{Exercise 1.7.2}
\end{homeworkProblem}
%----------------------------------------------------------------------------------------

%----------------------------------------------------------------------------------------
%       PROBLEM 8
%----------------------------------------------------------------------------------------
\newpage
\begin{homeworkProblem}
\matlabscript{exercise_01_07_02}{Exercise 1.7.2}
\end{homeworkProblem}
%----------------------------------------------------------------------------------------

%----------------------------------------------------------------------------------------
%       PROBLEM 9
%----------------------------------------------------------------------------------------
\newpage
\begin{homeworkProblem}
\matlabscript{exercise_01_07_03}{Exercise 1.7.3}
\end{homeworkProblem}
%----------------------------------------------------------------------------------------

%----------------------------------------------------------------------------------------
%       PROBLEM 10
%----------------------------------------------------------------------------------------
\newpage
\begin{homeworkProblem}
\matlabscript{exercise_01_08_01}{Exercise 1.8.1}
\end{homeworkProblem}
%----------------------------------------------------------------------------------------

%----------------------------------------------------------------------------------------
%       PROBLEM 11
%----------------------------------------------------------------------------------------
\newpage
\begin{homeworkProblem}
\matlabscript{exercise_01_08_02}{Exercise 1.8.2}
\end{homeworkProblem}
%----------------------------------------------------------------------------------------

\end{document}
