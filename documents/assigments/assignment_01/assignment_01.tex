
\documentclass{article}

\usepackage{fancyhdr} % Required for custom headers
\usepackage{lastpage} % Required to determine the last page for the footer
\usepackage{extramarks} % Required for headers and footers
\usepackage[usenames,dvipsnames]{color} % Required for custom colors
\usepackage{graphicx} % Required to insert images
\usepackage{listings} % Required for insertion of code
\usepackage{courier} % Required for the courier font
\usepackage{lipsum} % Used for inserting dummy 'Lorem ipsum' text into the template
\usepackage{caption}
\usepackage{subcaption}
\usepackage{mathtools}
\usepackage{amsmath}
\usepackage{esint}
\usepackage{physics}

\graphicspath{ {../../../datasets/images/chapter_02/} }

% Margins
\topmargin=-0.45in
\evensidemargin=0in
\oddsidemargin=0in
\textwidth=6.5in
\textheight=9.0in
\headsep=0.25in

\linespread{1.1} % Line spacing

% Set up the header and footer
\pagestyle{fancy}
\lhead{\hmwkAuthorName} % Top left header
\chead{\hmwkClass\ (\hmwkClassInstructor\ \hmwkClassTime): \hmwkTitle} % Top center head
\rhead{\firstxmark} % Top right header
\lfoot{\lastxmark} % Bottom left footer
\cfoot{} % Bottom center footer
\rfoot{Page\ \thepage\ of\ \protect\pageref{LastPage}} % Bottom right footer
\renewcommand\headrulewidth{0.4pt} % Size of the header rule
\renewcommand\footrulewidth{0.4pt} % Size of the footer rule

\setlength\parindent{0pt} % Removes all indentation from paragraphs

%----------------------------------------------------------------------------------------
%	CODE INCLUSION CONFIGURATION
%----------------------------------------------------------------------------------------

\definecolor{MyDarkGreen}{rgb}{0.0,0.4,0.0} 
\lstloadlanguages{Matlab}
\lstset{language=Matlab,
        frame=single,
        basicstyle=\small\ttfamily,
        keywordstyle=[1]\color{Blue}\bf,
        keywordstyle=[2]\color{Purple},
        keywordstyle=[3]\color{Blue}\underbar,
        identifierstyle=,
        commentstyle=\usefont{T1}{pcr}{m}{sl}\color{MyDarkGreen}\small, 
        stringstyle=\color{Purple},
        showstringspaces=false,
        tabsize=5, 
        morekeywords={rand},
        morekeywords=[2]{on, off, interp},
        morekeywords=[3]{test},
        morecomment=[l][\color{Blue}]{...},
        numbers=left,
        firstnumber=1,
        numberstyle=\tiny\color{Blue},
        stepnumber=5
}

% Creates a new command to include a perl script, the first parameter is the filename of the script (without .pl), the second parameter is the caption
\newcommand{\matlabscript}[2]{
\begin{itemize}
\item[]\lstinputlisting[caption=#2,label=#1]{#1.m}
\end{itemize}
}

%----------------------------------------------------------------------------------------
%	DOCUMENT STRUCTURE COMMANDS
%	Skip this unless you know what you're doing
%----------------------------------------------------------------------------------------

% Header and footer for when a page split occurs within a problem environment
\newcommand{\enterProblemHeader}[1]{
\nobreak\extramarks{#1}{#1 continued on next page\ldots}\nobreak
\nobreak\extramarks{#1 (continued)}{#1 continued on next page\ldots}\nobreak
}

% Header and footer for when a page split occurs between problem environments
\newcommand{\exitProblemHeader}[1]{
\nobreak\extramarks{#1 (continued)}{#1 continued on next page\ldots}\nobreak
\nobreak\extramarks{#1}{}\nobreak
}

\setcounter{secnumdepth}{0} % Removes default section numbers
\newcounter{homeworkProblemCounter} % Creates a counter to keep track of the number of problems

\newcommand{\homeworkProblemName}{}
\newenvironment{homeworkProblem}[1][Problem \arabic{homeworkProblemCounter}]{ % Makes a new environment called homeworkProblem which takes 1 argument (custom name) but the default is "Problem #"
\stepcounter{homeworkProblemCounter} % Increase counter for number of problems
\renewcommand{\homeworkProblemName}{#1} % Assign \homeworkProblemName the name of the problem
\section{\homeworkProblemName} % Make a section in the document with the custom problem count
\enterProblemHeader{\homeworkProblemName} % Header and footer within the environment
}{
\exitProblemHeader{\homeworkProblemName} % Header and footer after the environment
}

\newcommand{\problemAnswer}[1]{ % Defines the problem answer command with the content as the only argument
\noindent\framebox[\columnwidth][c]{\begin{minipage}{0.98\columnwidth}#1\end{minipage}} % Makes the box around the problem answer and puts the content inside
}

\newcommand{\homeworkSectionName}{}
\newenvironment{homeworkSection}[1]{ % New environment for sections within homework problems, takes 1 argument - the name of the section
\renewcommand{\homeworkSectionName}{#1} % Assign \homeworkSectionName to the name of the section from the environment argument
\subsection{\homeworkSectionName} % Make a subsection with the custom name of the subsection
\enterProblemHeader{\homeworkProblemName\ [\homeworkSectionName]} % Header and footer within the environment
}{
\enterProblemHeader{\homeworkProblemName} % Header and footer after the environment
}

%----------------------------------------------------------------------------------------
%	NAME AND CLASS SECTION
%----------------------------------------------------------------------------------------

\newcommand{\hmwkTitle}{Assignment\ \#1}
\newcommand{\hmwkDueDate}{Wednesday,\ September\ 27,\ 2017}
\newcommand{\hmwkClass}{3-D\ DIVISION} % Course/class
\newcommand{\hmwkClassTime}{15:00pm} % Class/lecture time
\newcommand{\hmwkClassInstructor}{leecw} % Teacher/lecturer
\newcommand{\hmwkAuthorName}{Tien Anh Nguyen} % Your name

%----------------------------------------------------------------------------------------
%	TITLE PAGE
%----------------------------------------------------------------------------------------

\title{
\vspace{2in}
\textmd{\textbf{\hmwkClass:\ \hmwkTitle}}\\
\normalsize\vspace{0.1in}\small{Due\ on\ \hmwkDueDate}\\
\vspace{0.1in}\large{\textit{\hmwkClassInstructor\ \hmwkClassTime}}
\vspace{3in}
}

\author{\textbf{\hmwkAuthorName}}
\date{} % Insert date here if you want it to appear below your name

%----------------------------------------------------------------------------------------

\begin{document}

\maketitle

%----------------------------------------------------------------------------------------
%	TABLE OF CONTENTS
%----------------------------------------------------------------------------------------

%\setcounter{tocdepth}{1} % Uncomment this line if you don't want subsections listed in the ToC

\newpage
\tableofcontents
\newpage

%----------------------------------------------------------------------------------------
%	PROBLEM 1
%----------------------------------------------------------------------------------------

% To have just one problem per page, simply put a \clearpage after each problem

\begin{homeworkProblem}
\subsection{Gram-Schmidt Process}
Given linearly independent vector \[x_1,x_2,...,x_k\], there exits mutually \
perpendicular vectors \[u_1,u_2,...,u_k\] with the same linear span. These may \
be constructed sequentially by setting
  \begin{equation}
   \begin{aligned}
    u_1 &= x_1\\
    u_2 &= x_2 - \frac{x'_2u_1}{u'_1u_1}{u_1}\\
    \qquad.\;\\
    \;\;.\;\\
    \;\;.\;\\
    u_k &= x_k - \frac{x'_ku_1}{u'_1u_1}u_1 - ... - \frac{(x'_ku_(k-1))}{u'_(k-1)u_(k-1)}u_(k-1)
   \end{aligned}
  \end{equation}
Gram-Schmidt process is used to find perpendicular vectors
\end{homeworkProblem}
%----------------------------------------------------------------------------------------

%----------------------------------------------------------------------------------------
%	PROBLEM 2
%----------------------------------------------------------------------------------------

% To have just one problem per page, simply put a \clearpage after each problem
\newpage
\begin{homeworkProblem}
\subsection{Gauss-Newton Algorithm}
According to Wikipedia, the Gauss-Newton algorithm is used to solve non-linear least\
squares problems.
The goal is to model a set of N data points \[{(x_i, y_i), (i = 1, ..., N)}\] \
by a non-linear function.
\begin{equation}
y = f(x, a_1, ..., a_M) = f(x, a)
\end{equation}
with a set of M model parameters \[a = [a_1,...,a_M]^T\], so that the sum of\
 squared error is miniminzed:
\begin{equation}
\varepsilon(a) = \sum_{i=1}^{N}r_{i}^2 = \sum_{i=1}^{N}[y_i - f(x_i,a)]^2
\end{equation}
We defined:
\begin{equation}
\begin{aligned}
f_i(a) &= f(x_i,a)\\
r_i &= y_i - f(x_i, a) = y_i - f_i(a)
\end{aligned}
\end{equation}
Therefore, we have:
\begin{equation} \label{eq:5}
\varepsilon(a) = \sum_{i=1}^{N}[y_i - f(x_i,a)]^2 = \sum_{i=1}^{N}[y_i - f_i(a)]^2 = \sum_{i=1}^{N}r_{i}^2 
\end{equation}
Applying vector forms, the equation (\ref{eq:5}) can be rewritten as:

\begin{equation}
\begin{aligned}
r = [
    \begin{matrix}
        r_{1}\\
        \hdotsfor{1}\\
        \hdotsfor{1}\\
        \hdotsfor{1}\\
        r_{N}
    \end{matrix}
]
  =
[
    \begin{matrix}
        y_{1} - f_{1}(a)\\
        \hdotsfor{1}\\
        \hdotsfor{1}\\
        \hdotsfor{1}\\
        y_{N} - f_{N}(a)\\
    \end{matrix}
]
  =
[
    \begin{matrix}
        y_{1}\\
        \hdotsfor{1}\\
        \hdotsfor{1}\\
        \hdotsfor{1}\\
        y_{N}\\
    \end{matrix}
]
  -
[
    \begin{matrix}
        f_{1}(a)\\
        \hdotsfor{1}\\
        \hdotsfor{1}\\
        \hdotsfor{1}\\
        f_{N}(a)\\
    \end{matrix}
]
  =
    y - f(a)
\end{aligned}
\end{equation}
where
\begin{equation}
\begin{aligned}
y = [y_1, ..., y_N]^T\\
f(a) = [f_1(a), ...,f_N(a)]^T
\end{aligned}
\end{equation}
Therefore,
\begin{equation}\label{eq:8}
\varepsilon(a) = \sum_{i=1}^{N}r_{i}^2 = ||r||^2 = ||y - f(a)||^2
\end{equation}
The epsilon value in the equation (\ref{eq:8}) is minimum if it satisfy the equation that \
the gradient vector is equal to zero. Hence,

\begin{equation}
\dv{\varepsilon(a)}{a} = \dv{||y - f(a)||^2}{a} = -2J^T(y - f(a)) = 0
\end{equation}

Where J is the Jacobian matrix.

However, It is really hard to find a exact solution for a, so we are going to \
find a approximate one. We use the following iteration:
\begin{equation}
a_{n+1} = a_{n} + \Delta a
\end{equation}
According to the Taylor expansion of the function 
\begin{equation}
f_i(a_{n+1}) \quad at \quad a_n
\end{equation}
 we have:
\begin{equation}
f(a_{n+1}) \approx f(a_n) + J\Delta a
\end{equation}

The Gauss-Newton algorithm is quite complex. It require a lot of knowledge about\
 Algebra, Jacobian matrix, Taylor expansion and Newton's method. I are going to \
obtain those knowledge as soon as possible to upgrade this report.

I am going to show the result of Gauss-Newton method is:
\begin{equation}
a_{n+1} = a_n + \Delta a = a_n + J^-(y - f(a_n))
\end{equation}
\end{homeworkProblem}
%----------------------------------------------------------------------------------------
\end{document}
