
\documentclass{article}

\usepackage{fancyhdr} % Required for custom headers
\usepackage{lastpage} % Required to determine the last page for the footer
\usepackage{extramarks} % Required for headers and footers
\usepackage[usenames,dvipsnames]{color} % Required for custom colors
\usepackage{graphicx} % Required to insert images
\usepackage{listings} % Required for insertion of code
\usepackage{courier} % Required for the courier font
\usepackage{lipsum} % Used for inserting dummy 'Lorem ipsum' text into the template
\usepackage{caption}
\usepackage{subcaption}
\usepackage{mathtools}
\usepackage{amsmath}
\usepackage{esint}
\usepackage{physics}
\usepackage{color}
\usepackage{listings}
\usepackage{pythonhighlight}

\graphicspath{ {../../../datasets/images/chapter_02/} }

% Margins
\topmargin=-0.45in
\evensidemargin=0in
\oddsidemargin=0in
\textwidth=6.5in
\textheight=9.0in
\headsep=0.25in

\linespread{1.1} % Line spacing

% Set up the header and footer
\pagestyle{fancy}
\lhead{\hmwkAuthorName} % Top left header
\chead{\hmwkClass\ (\hmwkClassInstructor\ \hmwkClassTime): \hmwkTitle} % Top center head
\rhead{\firstxmark} % Top right header
\lfoot{\lastxmark} % Bottom left footer
\cfoot{} % Bottom center footer
\rfoot{Page\ \thepage\ of\ \protect\pageref{LastPage}} % Bottom right footer
\renewcommand\headrulewidth{0.4pt} % Size of the header rule
\renewcommand\footrulewidth{0.4pt} % Size of the footer rule

\setlength\parindent{0pt} % Removes all indentation from paragraphs

%----------------------------------------------------------------------------------------
%	CODE INCLUSION CONFIGURATION
%----------------------------------------------------------------------------------------

\definecolor{MyDarkGreen}{rgb}{0.0,0.4,0.0} 
\lstloadlanguages{Matlab}
\lstset{language=Matlab,
        frame=single,
        basicstyle=\small\ttfamily,
        keywordstyle=[1]\color{Blue}\bf,
        keywordstyle=[2]\color{Purple},
        keywordstyle=[3]\color{Blue}\underbar,
        identifierstyle=,
        commentstyle=\usefont{T1}{pcr}{m}{sl}\color{MyDarkGreen}\small, 
        stringstyle=\color{Purple},
        showstringspaces=false,
        tabsize=5, 
        morekeywords={rand},
        morekeywords=[2]{on, off, interp},
        morekeywords=[3]{test},
        morecomment=[l][\color{Blue}]{...},
        numbers=left,
        firstnumber=1,
        numberstyle=\tiny\color{Blue},
        stepnumber=5
}

% Creates a new command to include a perl script, the first parameter is the filename of the script (without .pl), the second parameter is the caption
\newcommand{\matlabscript}[2]{
\begin{itemize}
\item[]\lstinputlisting[caption=#2,label=#1]{#1.m}
\end{itemize}
}

%----------------------------------------------------------------------------------------
%	DOCUMENT STRUCTURE COMMANDS
%	Skip this unless you know what you're doing
%----------------------------------------------------------------------------------------

% Header and footer for when a page split occurs within a problem environment
\newcommand{\enterProblemHeader}[1]{
\nobreak\extramarks{#1}{#1 continued on next page\ldots}\nobreak
\nobreak\extramarks{#1 (continued)}{#1 continued on next page\ldots}\nobreak
}

% Header and footer for when a page split occurs between problem environments
\newcommand{\exitProblemHeader}[1]{
\nobreak\extramarks{#1 (continued)}{#1 continued on next page\ldots}\nobreak
\nobreak\extramarks{#1}{}\nobreak
}

\setcounter{secnumdepth}{0} % Removes default section numbers
\newcounter{homeworkProblemCounter} % Creates a counter to keep track of the number of problems

\newcommand{\homeworkProblemName}{}
\newenvironment{homeworkProblem}[1][Problem \arabic{homeworkProblemCounter}]{ % Makes a new environment called homeworkProblem which takes 1 argument (custom name) but the default is "Problem #"
\stepcounter{homeworkProblemCounter} % Increase counter for number of problems
\renewcommand{\homeworkProblemName}{#1} % Assign \homeworkProblemName the name of the problem
\section{\homeworkProblemName} % Make a section in the document with the custom problem count
\enterProblemHeader{\homeworkProblemName} % Header and footer within the environment
}{
\exitProblemHeader{\homeworkProblemName} % Header and footer after the environment
}

\newcommand{\problemAnswer}[1]{ % Defines the problem answer command with the content as the only argument
\noindent\framebox[\columnwidth][c]{\begin{minipage}{0.98\columnwidth}#1\end{minipage}} % Makes the box around the problem answer and puts the content inside
}

\newcommand{\homeworkSectionName}{}
\newenvironment{homeworkSection}[1]{ % New environment for sections within homework problems, takes 1 argument - the name of the section
\renewcommand{\homeworkSectionName}{#1} % Assign \homeworkSectionName to the name of the section from the environment argument
\subsection{\homeworkSectionName} % Make a subsection with the custom name of the subsection
\enterProblemHeader{\homeworkProblemName\ [\homeworkSectionName]} % Header and footer within the environment
}{
\enterProblemHeader{\homeworkProblemName} % Header and footer after the environment
}

%----------------------------------------------------------------------------------------
%	NAME AND CLASS SECTION
%----------------------------------------------------------------------------------------

\newcommand{\hmwkTitle}{Assignment\ \#2}
\newcommand{\hmwkDueDate}{Wednesday,\ September\ 27,\ 2017}
\newcommand{\hmwkClass}{3-D\ DIVISION} % Course/class
\newcommand{\hmwkClassTime}{15:00pm} % Class/lecture time
\newcommand{\hmwkClassInstructor}{leecw} % Teacher/lecturer
\newcommand{\hmwkAuthorName}{Tien Anh Nguyen} % Your name

%----------------------------------------------------------------------------------------
%	TITLE PAGE
%----------------------------------------------------------------------------------------

\title{
\vspace{2in}
\textmd{\textbf{\hmwkClass:\ \hmwkTitle}}\\
\normalsize\vspace{0.1in}\small{Due\ on\ \hmwkDueDate}\\
\vspace{0.1in}\large{\textit{\hmwkClassInstructor\ \hmwkClassTime}}
\vspace{3in}
}

\author{\textbf{\hmwkAuthorName}}
\date{} % Insert date here if you want it to appear below your name

%----------------------------------------------------------------------------------------

\begin{document}

\maketitle

%----------------------------------------------------------------------------------------
%	TABLE OF CONTENTS
%----------------------------------------------------------------------------------------

%\setcounter{tocdepth}{1} % Uncomment this line if you don't want subsections listed in the ToC

\newpage
\tableofcontents
\newpage

%----------------------------------------------------------------------------------------
%	PROBLEM 1
%----------------------------------------------------------------------------------------

% To have just one problem per page, simply put a \clearpage after each problem

\begin{homeworkProblem}

Consider two images $x_1$, $x_2$ of the same point p from two camera positions with \
relative pose (R, T), where $R \in SO(3)$ is the relative orientation and \
 $T \in \!R^3$ is the relative position. Then $x_1,\ x_2$ satisfy
 \begin{equation}
 \label{essential constraint}
   \langle x_2, T \times Rx_1\rangle = 0,\ \ or\ \ x_2^T\hat T R x_1 = 0
 \end{equation}

 \begin{equation}
 \label{essential matrix}
   E = \hat TR\ \in\ \!R^{3x3}
 \end{equation}

The matrix \textbf{E} is called the \textbf{essential matrix}. The epipolar constraint \
[\ref{essential constraint}] is also called the \textbf{essential constraint}.

\textbf{Theorem 5.5(Characterization of the essential matrix)}. A nonzero matrix $E\ \in\ \!R^{3x3}$ \
is an essential matrix if and only if \textbf{E} has a singular value decomposition (SVD).
 \begin{equation}
   E = U \Sigma V^T\ with \ \Sigma = diag{\sigma,\sigma,0}
 \end{equation}

Because E has a singular value decomposition, we can rewrite E as $E = U \Sigma V^T$.
We need to prove that $E = \hat TR$.
By definition, for any essential matrix \textbf{E}, there exists $(R,T)$ with $R \in SO(3)$,\
such that $\hat TR=E$. SO(3) is a Lie group. For T, there exists a rotation matrix $R_0$ such \
that $R_0T=[0,0,||T||]^T$. Define $a = R_0T \ \in \!R^3$.
From \textbf{Lemma 5.4}, for a vector $T\ \in\ \!R^3$ and a matrix $K\ \in\ \!R^{3x3}$, \
if $det(K)\ =\ +1$ and $T^{'}\ =\ KT$, then $\hat T\ =\ K^T \hat{T'}K$.
Define $a = R_0T \in \!R^3$. Since $det(R_0) = 1$, we know that $\hat T = R_0^T \hat a R_0$.
Then $EE^T = \hat T R R^T \hat T^T = \hat T \hat T^T = R_0^T \hat a \hat a^T R_0$.
We use the matrices
 \begin{equation}
   \hat a = 
      \begin{bmatrix}
         0 & -||T|| & 0 \\
         ||T||  &  0 & 0 \\
         0 &  0 & 1
      \end{bmatrix}
   \hat a^T = 
     \begin{bmatrix}
          0 & ||T|| & 0 \\
         -||T|| & 0 & 0 \\
          0 & 0 & 0
     \end{bmatrix}
 \end{equation}

 The singular values of the essential matrix $E=\hat TR$ are $(||T||, ||T||, 0)$. We need to prove that \
U, V have determinant is +1.
 \begin{equation}
   E = \hat TR = R_0^T \hat a R_0 R
 \end{equation}
 \begin{equation}
   R_Z(\frac{+\pi}{2}) = 
      \begin{bmatrix}
         0 & -1 & 0 \\
         1 &  0 & 0 \\
         0 &  0 & 1
      \end{bmatrix}
 \end{equation}

$R_Z(\frac{+\pi}{2})$ is the matrix that represents a rotation around the Z-axis by an angle of $\frac{+\pi}{2}$.

 \begin{equation}
    \hat a = R_Z(\frac{+\pi}{2})R_Z^T(\frac{+\pi}{2})\hat a = R_Z(\frac{+\pi}{2})diag\{||T||, ||T||, 0\}
 \end{equation}

\begin{equation}
  (\hat T_1, R_1) = (U R_Z(\frac{+\pi}{2}) \Sigma U^T, U R_Z^T(\frac{+\pi}{2}) V^T)\\
\end{equation}
\begin{equation}
  (\hat T_2, R_2) = (U R_Z(\frac{-\pi}{2}) \Sigma U^T, U R_Z^T(\frac{-\pi}{2}) V^T)
\end{equation}

 \begin{equation}
   \hat T_1 R_1 = \hat T_2 R_2 = E
 \end{equation}

Thus, E is an essential matrix.
\end{homeworkProblem}

%----------------------------------------------------------------------------------------

\end{document}
