
\documentclass{article}

\usepackage{fancyhdr} % Required for custom headers
\usepackage{lastpage} % Required to determine the last page for the footer
\usepackage{extramarks} % Required for headers and footers
\usepackage[usenames,dvipsnames]{color} % Required for custom colors
\usepackage{graphicx} % Required to insert images
\usepackage{listings} % Required for insertion of code
\usepackage{courier} % Required for the courier font
\usepackage{lipsum} % Used for inserting dummy 'Lorem ipsum' text into the template
\usepackage{caption}
\usepackage{subcaption}
\usepackage{amsmath}

\graphicspath{ {../../../datasets/images/chapter_02/} }

% Margins
\topmargin=-0.45in
\evensidemargin=0in
\oddsidemargin=0in
\textwidth=6.5in
\textheight=9.0in
\headsep=0.25in

\linespread{1.1} % Line spacing

% Set up the header and footer
\pagestyle{fancy}
\lhead{\hmwkAuthorName} % Top left header
\chead{\hmwkClass\ (\hmwkClassInstructor\ \hmwkClassTime): \hmwkTitle} % Top center head
\rhead{\firstxmark} % Top right header
\lfoot{\lastxmark} % Bottom left footer
\cfoot{} % Bottom center footer
\rfoot{Page\ \thepage\ of\ \protect\pageref{LastPage}} % Bottom right footer
\renewcommand\headrulewidth{0.4pt} % Size of the header rule
\renewcommand\footrulewidth{0.4pt} % Size of the footer rule

\setlength\parindent{0pt} % Removes all indentation from paragraphs

%----------------------------------------------------------------------------------------
%	CODE INCLUSION CONFIGURATION
%----------------------------------------------------------------------------------------

\definecolor{MyDarkGreen}{rgb}{0.0,0.4,0.0} 
\lstloadlanguages{Matlab}
\lstset{language=Matlab,
        frame=single,
        basicstyle=\small\ttfamily,
        keywordstyle=[1]\color{Blue}\bf,
        keywordstyle=[2]\color{Purple},
        keywordstyle=[3]\color{Blue}\underbar,
        identifierstyle=,
        commentstyle=\usefont{T1}{pcr}{m}{sl}\color{MyDarkGreen}\small, 
        stringstyle=\color{Purple},
        showstringspaces=false,
        tabsize=5, 
        morekeywords={rand},
        morekeywords=[2]{on, off, interp},
        morekeywords=[3]{test},
        morecomment=[l][\color{Blue}]{...},
        numbers=left,
        firstnumber=1,
        numberstyle=\tiny\color{Blue},
        stepnumber=5
}

% Creates a new command to include a perl script, the first parameter is the filename of the script (without .pl), the second parameter is the caption
\newcommand{\matlabscript}[2]{
\begin{itemize}
\item[]\lstinputlisting[caption=#2,label=#1]{#1.m}
\end{itemize}
}

%----------------------------------------------------------------------------------------
%	DOCUMENT STRUCTURE COMMANDS
%	Skip this unless you know what you're doing
%----------------------------------------------------------------------------------------

% Header and footer for when a page split occurs within a problem environment
\newcommand{\enterProblemHeader}[1]{
\nobreak\extramarks{#1}{#1 continued on next page\ldots}\nobreak
\nobreak\extramarks{#1 (continued)}{#1 continued on next page\ldots}\nobreak
}

% Header and footer for when a page split occurs between problem environments
\newcommand{\exitProblemHeader}[1]{
\nobreak\extramarks{#1 (continued)}{#1 continued on next page\ldots}\nobreak
\nobreak\extramarks{#1}{}\nobreak
}

\setcounter{secnumdepth}{0} % Removes default section numbers
\newcounter{homeworkProblemCounter} % Creates a counter to keep track of the number of problems

\newcommand{\homeworkProblemName}{}
\newenvironment{homeworkProblem}[1][Problem \arabic{homeworkProblemCounter}]{ % Makes a new environment called homeworkProblem which takes 1 argument (custom name) but the default is "Problem #"
\stepcounter{homeworkProblemCounter} % Increase counter for number of problems
\renewcommand{\homeworkProblemName}{#1} % Assign \homeworkProblemName the name of the problem
\section{\homeworkProblemName} % Make a section in the document with the custom problem count
\enterProblemHeader{\homeworkProblemName} % Header and footer within the environment
}{
\exitProblemHeader{\homeworkProblemName} % Header and footer after the environment
}

\newcommand{\problemAnswer}[1]{ % Defines the problem answer command with the content as the only argument
\noindent\framebox[\columnwidth][c]{\begin{minipage}{0.98\columnwidth}#1\end{minipage}} % Makes the box around the problem answer and puts the content inside
}

\newcommand{\homeworkSectionName}{}
\newenvironment{homeworkSection}[1]{ % New environment for sections within homework problems, takes 1 argument - the name of the section
\renewcommand{\homeworkSectionName}{#1} % Assign \homeworkSectionName to the name of the section from the environment argument
\subsection{\homeworkSectionName} % Make a subsection with the custom name of the subsection
\enterProblemHeader{\homeworkProblemName\ [\homeworkSectionName]} % Header and footer within the environment
}{
\enterProblemHeader{\homeworkProblemName} % Header and footer after the environment
}

%----------------------------------------------------------------------------------------
%	NAME AND CLASS SECTION
%----------------------------------------------------------------------------------------

\newcommand{\hmwkTitle}{Assignment\ \#1}
\newcommand{\hmwkDueDate}{Tuesday,\ April\ 08,\ 2018}
\newcommand{\hmwkClass}{Topics In Deep Learning} % Course/class
\newcommand{\hmwkClassTime}{15:00am} % Class/lecture time
\newcommand{\hmwkClassInstructor}{shkim} % Teacher/lecturer
\newcommand{\hmwkAuthorName}{Tien Anh Nguyen} % Your name

%----------------------------------------------------------------------------------------
%	TITLE PAGE
%----------------------------------------------------------------------------------------

\title{
\vspace{2in}
\textmd{\textbf{\hmwkClass:\ \hmwkTitle}}\\
\normalsize\vspace{0.1in}\small{Due\ on\ \hmwkDueDate}\\
\vspace{0.1in}\large{\textit{\hmwkClassInstructor\ \hmwkClassTime}}
\vspace{3in}
}

\author{\textbf{\hmwkAuthorName}}
\date{} % Insert date here if you want it to appear below your name

%----------------------------------------------------------------------------------------

\begin{document}

\maketitle

%----------------------------------------------------------------------------------------
%	TABLE OF CONTENTS
%----------------------------------------------------------------------------------------

%\setcounter{tocdepth}{1} % Uncomment this line if you don't want subsections listed in the ToC

\newpage
\tableofcontents
\newpage

%----------------------------------------------------------------------------------------


%-------------------------------------------------------------------------------
%       PROBLEM 1
%-------------------------------------------------------------------------------
\newpage
\begin{homeworkProblem}
\subsection{Summary of the paper Lecun, Y., Bengio, Y., \& Hinton, G. (2015). Deep learning.Nature, 521(7553), 436–444.}

\subsubsection{Abstraction}
Deep learning allows computational models that are composed of multiple processing layers to learn representations of data with multiple levels of abstraction.

Deep learning is widely applied in numerous domain such as Speech Recognition, Visual Object Recognition, Object Detection, Drug Discovery, Genomics, etc...

The ability of tradition machine learning is limited for processing natural data in their raw form.

Coming up with features is difficult, time-consuming task. It is also required 
expert knowledge. Researchers and developers need to spend a lot of time to tunning features of data.

Neural network is used for doing a complex transformation. With the composition of enough such transformation, very complex functions can be learned.

\subsubsection{Supervised Learning}
The procedure of supervised learning requires: collecting dataset, labeling, training (errors, tuning parameters, gradient descent) and  testing.
A objective function is computed to measure the distance between the output scores and the desired pattern of scores. The machine learning system modifies its internal parameters which are real number to produce this scores. These parameter is called weights. A typical deep learning system can contains hundreds of millions of these adjustable weights.

\subsubsection{Back-propagation to Train Multilayer Architectures}
The back-propagation procedure to compute the gradient of an objective function with respect to the weights of a multilayer stack modules is nothing more than a practical application of the chain rule for derivatives.

The derivative (gradient) of the objective with respect to the input of a module can be computed by working backwards from the gradient with respect to the output of that module.

\subsubsection{Convolutional Neural Networks}
Convolutional neural networks are designed to process data that come in the form of multiple arrays. The architecture of a typical ConvNet is structured as a series of stages. The first few stages are composed of two types of layers: convolutional layers and pooling layers. The pooling layer merges similar features into one. Units in a convolutional layer are organized in feature maps, within which each unit is connected to local patches in the feature maps of the previous layer through a set of weights called a filter bank.

\subsubsection{Recurrent Neural Networks}
It is better to use RNNs for tasks that involve sequential inputs, such as speech and language. RNNs are trained by back-propagation.
The RNNs network have 2 specific types are: Long short-term memory (LSTM)
 and GRU.
A recurrent neural network and the unfolding in time of the computation involved in its forward computation. The artificial neurons get inputs from other neurons at previous time steps.

\subsubsection{The future of deep learning}
Unsupervised learning is the future of deep learning. Human and animal learning is largely unsupervised, we expect unsupervised learning to become far more important in the longer term. 

\end{homeworkProblem}
%-------------------------------------------------------------------------------

%-------------------------------------------------------------------------------
%       PROBLEM 2
%-------------------------------------------------------------------------------
\newpage
\begin{homeworkProblem}
    \subsection{Using neural nets to recognize handwritten digits}
    \subsubsection{Introduction}
    Human can easily recognize the digit numbers from 0 to 9. However, it is very difficult for machines to do the same task.
    
    \subsubsection{Neural Networks}
    Neural networks approach the problem in a different way. The idea is to take a large number of handwritten digits, known as training examples, and then develop a system which can learn from those training examples. The neural network automatically infer rules from the examples to recognize the digit numbers.
    
    \subsubsection{Perceptrons}
    A perceptron takes several inputs, $x_1$, $x_2$, ..., and produces a single or multiple outputs. The perception is a machine which makes decisions by weighing up evidence.
    To represent a complex desired decision-making model, we use multiple layer perceptrons. The first layer of perceptrons weight the input evidence. The second layer is making a decision by weighing up the results from the first layer. In this way a perceptron in the second layer can make a decision at a more complex and more abstract level than perceptrons in the first layer. And even more complex decisions can be made by the perceptron in the third layer.
    
    \subsubsection{Sigmoid Neurons}
    A sigmoid neuron has input $x_1$, $x_2$, ... which are real number in [0, 1] interval. The activation is the sigmoid function
    \subsubsection{The architecture of neural networks}
    The input layer has $64 \times 64 = 4096$ input neurons. The output layer will contains just a single neuron, with output values less than 0.5 indicating "input image is not a specific number", and values greater than 0.5 indicating "input image is the specific number".
    
    \subsubsection{A simple network to classify handwritten digits}
    The architecture contains three layers: one hidden layer. The input image has $28 \times 28$ pixel. Therefore, the input layer contains $28 \times 28 = 784$ neurons. The output layer has 10 neurons for digit numbers from 0 to 9. The number of neurons of the hidden layer is based on experiments.
    \subsubsection{Learning with gradient descent}
    We are going to find an algorithm which has ability to find weights and biases so that the output of the neural network approximate the desired values for all training input $x$. Let define a cost function
    \begin{equation}
        C(w, b) = \frac{1}{2n}\sum_{x}||y(x) - a||^2
    \end{equation}
    At the point the cost function reaches global minimum values, we find the suitable weights and biases for our model.
\end{homeworkProblem}
%-------------------------------------------------------------------------------


\end{document}
